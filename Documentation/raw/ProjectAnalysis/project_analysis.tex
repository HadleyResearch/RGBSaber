% File: project_overview.tex
% Description: TeX file to generate project overview template
% Author: George Hadley
% Website: http://nbitwonder.com
% Notes:
% 1) This document is written using the LaTeX typesetting language. For a beginner's
%	guide to the LaTeX typesetting language, consult http://en.wikibooks.org/wiki/LaTeX
% 2) This document needs to be compiled using pdfLaTeX. It is not compatible with pdfTeX 
%	at the present time.
% 3) This document utilizes the nbwheader function, located in the doc_header.tex file.
%	This file is located at /path-to-documentation-templates/lbr/doc_header.tex.
% Version: 0.1
% Last Modified: 1-04-2010

\documentclass[12pt,letterpaper,onecolumn]{article}
\usepackage{graphicx}
\usepackage{float}
\usepackage{subfig}
\usepackage{tikz}
\usepackage{fancyhdr}
\usepackage{verbatim}
\usepackage{kpfonts}
\usepackage{fullpage}
\usepackage{hyperref}

%Pathway to global documentation library functions
%Modify to /path-to-documentation-templates/lbr
\newcommand{\globallbr}{../lbr}

%Page layout settings
\setlength{\voffset}{-10pt}
\setlength{\headsep}{20pt}
\setlength{\headheight}{15pt}
\setlength{\topmargin}{-20pt}

\begin{comment}
  Hyperref settings: settings for the hyperref hyperlink package.
  For a more detailed listing of available settings, consult 
  	http://en.wikibooks.org/wiki/LaTeX/Hyperlinks#Customization

  IMPORTANT: For your document, modify the pdftitle, pdfauthor, pdfsubject,
	and pdfkeywords options to tailor to your document
\end{comment}
\hypersetup{
	bookmarks=true, 					%Enable pdf bookmarks
	pdfborder={0,0,0},					%Disable borders around links
	pdftitle={Project Overview (template)},		%Name of PDF document
	pdfauthor={George Hadley},				%Author of PDF document
	pdfsubject={Embedded Electronics},		%Subject of PDF document
	pdfkeywords={diy,electronics,nbitwonder},	%Keywords for PDF document
	colorlinks={true},					%Enable colored links
	linkcolor=red,						%Internal link color
	citecolor=green,						%Citation link color
	filecolor=blue,						%File link color
	urlcolor=blue						%URL link color
}

% File: doc_header.tex
% Description: TeX command used to develop custom documentation header used in Open
%	 Documentation System
% Author: George Hadley
% Website: http://nbitwonder.com
% Notes: 
% 	1) Derived from LaTeX example 'TeXblog: Fancy chapter headings with TikZ' found at
%		http://texblog.net/latex-archive/layout/fancy-chapter-tikz/
%	2) For additional help with the LaTeX package, consult the WikiBook found at
%		http://en.wikibooks.org/wiki/LaTeX
%	3) This file needs to be compiled with pdfLaTeX (does not appear to be compatible with
%		pdfTeX at the present time)
% Version: 0.1
% Last Modified: 12-28-2010

%Variable Declarations (modify these to tweak the header on your document)
\newcommand{\headery}{-3cm}					%y-offset of header (bottom left corner of header)
\newcommand{\headercolor}{black}				%Background color of header
\newcommand{\headerxstart}{.1\paperwidth}			%Starting x-coordinate for header box
\newcommand{\headerystart}{0cm}				%Starting y-coordinate for header box
\newcommand{\headerxend}{.9\paperwidth}			%Ending x-coordinate for header box
\newcommand{\headeryend}{2cm}				%Ending y-coordinate for header box
\newcommand{\logoxstart}{.1\paperwidth}			%Starting x-coordinate for logo
\newcommand{\logoystart}{1cm}					%Starting y-coordinate for logo
\newcommand{\logolink}{http://nbitwonder.com}		%URL for website, etc.
\newcommand{\logoheight}{40pt}					%Height of logo image
\newcommand{\logoloc}{../lbr/img/NBitWonderLogo_wTm.png}	%Path to logo image
\newcommand{\docboxxstart}{.5\paperwidth}		%Starting x-coordinate of documentation box
\newcommand{\docboxystart}{1cm}				%Starting y-coordinate of documentation box
\newcommand{\docboxwidth}{.38\paperwidth}		%Width of documentation box
\newcommand{\docboxtextcolor}{green}			%Text color in documentation box

% Command: \nbwheader
% Description: Creates a documentation header for use in the Open Documentation System templates
% Usage: \nbwheader{DocumentName}{ProjectName}{ProjectVersion} where
%	DocumentName is the name of the document (e.g. Project Notebook, User Manual, Bug Tracker, etc.
%	ProjectName is the name of the project the documentation is about
%	ProjectVersion is the current version of the project
\newcommand{\nbwheader}[3]
{
    \begin{tikzpicture}[remember picture,overlay]
    \node[yshift=\headery] at (current page.north west)
    {
        \begin{tikzpicture}[remember picture, overlay]
        \draw[fill=\headercolor] (\headerxstart,\headerystart) rectangle
            (\headerxend,\headeryend);
        \node[anchor=west,xshift=\logoxstart,yshift=\logoystart,rectangle]
        {\href{\logolink}{\includegraphics[height=\logoheight]{\logoloc}}};
        \node[anchor=west,xshift =\docboxxstart,yshift=\docboxystart,rectangle]
        {
            \begin{minipage}{\docboxwidth}
	   \begin{flushright}
	   \normalsize\textcolor{\docboxtextcolor}{\textsc{
	   \linespread{2}
	   #1 \\
	   Project: #2 \\
	   Version: #3 \\
            }}
	   \end{flushright}
            \end{minipage}};
        \end{tikzpicture}
        };
    \end{tikzpicture}
}

% File: aliases.tex
% Description: TeX library aliases used in open documentation templates
% Author: George Hadley
% Website: http://nbitwonder.com
% Notes: 
%	1) For additional help with the LaTeX package, consult the WikiBook found at
%		http://en.wikibooks.org/wiki/LaTeX
%	2) This file needs to be compiled with pdfLaTeX (does not appear to be compatible with
%		pdfTeX at the present time)
% Version: 0.1
% Last Modified: 1-04-2010
\newcommand{\ohm}{$\Omega$}
\newcommand{\uF}{$\mu$F}

%Modify these to reflect your documentation
\newcommand{\documentationtype}{Project Overview}
\newcommand{\projectname}{RGBSaber }
\newcommand{\projectversion}{1}

%Header/Footer Definitions
\pagestyle{fancy}
\lhead{ }
\chead{ }
\rhead{\projectname  v\projectversion \documentationtype}
\lfoot{\href{http://nbitwonder.com}{http://nbitwonder.com} }
\cfoot{\thepage}
\rfoot{\copyright 2011 NBitWonder, LLC }

\begin{document}
\thispagestyle{plain}
% Insert title page or header here
\nbwheader{\documentationtype}{\projectname}{\projectversion}

%Uncomment to add Table of Contents to Project Overview
\tableofcontents
\newpage

\section[Project Description]{Project Description}
% Section: Project Description
% Description: A brief description of what your project is and what it will be used for
The RGBSaber is an open source RGB Luxeon LED driver. The driver allows for PWM modulation of an RGB
luxeon LED, allowing the color of the lightsaber blade to be adjusted to nearly any color in the RGB color spectrum. The current version of the lightsaber firmware supports 24-bit color, allowing for 16,777,216 different color combinations. The RGBSaber project, and all other NBitWonder projects, are made available under a \href{http://creativecommons.org/licenses/by-sa/3.0/}{Creative Commons 3.0 BY-SA license}.

\section[Schematics]{Schematics and Block Diagrams}
% Section: Schematics and Block Diagrams
% Description: Include any relevant high level schematics and block diagrams in this section

\section[Project Roadmap]{Project Roadmap}
% Section: Project Roadmap
% Description: Detail improvements and upgrades to be considered for future versions 
%	of the project
The RGBSaber system has considerable upgrade potential over the course of the project. A rough
project roadmap looks something like this:

\underline{\textbf{Control Board}}
\begin{enumerate}
\item\textbf{Version 1:} Basic RGBSaber system, capable of full RGB color control.
\item\textbf{Version 2:} Modular system, featuring hardware plugins for sound. Integration of
	sound with RGB color control
\item\textbf{Version 3:} Upgraded lightsaber transistor drive capabilities.
\end{enumerate}

\textbf{\underline{Peripheral Board}}
\begin{enumerate}
\item\textbf{Version 1:} Simple functional peripheral board.
\item\textbf{Version 2:} Improved peripheral board with touch interface
\end{enumerate}

\section[Project Analysis]{Project Analysis}
% Section: Project Analysis
% Description: Analysis of project compared to similar/competing projects, products, 
%	and patents. Divided into subsections (below)

\subsection[Open Source]{Open Source Project Analysis}
% Subsection: Open Source Analysis
% Description: Analysis of project compared to similar open source projects.
To date, no other known open source RGB lightsaber systems appear to exist. That said, there are a number 
of boards online that are designed to do RGB color mixing.

\subsubsection[Dangerous Prototypes]{Dangerous Prototypes USB RGB Color Changer}
Dangerous Prototypes produced a \href{http://dangerousprototypes.com/2010/01/21/usb-rgb-color-changer-build/}{USB RGB color changer} some time ago. This design was only ever in the prototype stage,
and is not in production. Additionally, the boards are far too large to fit into a lightsaber form factor and
use through-hole parts. As such, this design is not seen as a strong competitor to the RGB Lightsaber
design.

\subsubsection[Macetech]{MaceTech ShiftBrite Project}
Open source company Macetech developed and manufactures an RGB color mixing system known as the
\href{http://macetech.com/blog/?q=node/23}{ShiftBrite}. The ShiftBrite RGB system is theoretically small
enough to fit inside of a lightsaber hilt environment. However, the LEDs and the driving circuitry for the
ShiftBrite system are not bright and powerful enough to see use in a lightsaber system.

\subsection[Commercial]{Commercial Product Analysis}
% Subsection: Commercial Product Analysis
% Description: Analysis of project compared to similar commercial products
\subsubsection[Sparkfun]{Sparkfun Electronics RGB Triple Play}
In late 2010, Sparkfun electronics released the \href{http://www.sparkfun.com/products/9738}{RGB Triple Play} and an associated \href{http://www.sparkfun.com/products/9834}{driver}.
Coupled with a triple LED lens, these products certainly hit closer to the RGBSaber system. However, at the
time of this writing, these products are still too bulky to be used in a lightsaber environment, and also vastly
overpriced (costing approximately 70 dollars for the LED and drivers alone, without any control electronics).
As such, Sparkfun's products are not seen as a particular threat to the RGBSaber project.

\subsection[Plecter Labs]{Plecter Labs Crystal Focus System}
Lightsaber company Plecter Labs produces a lightsaber core system known as \href{http://www.plecterlabs.com/catalog/product_info.php?cPath=23&products_id=149}{Crystal Focus}
which is an extremely strong competitor to the RGBSaber project. The board is extremely high end, and 
features many advanced features, particularly sound and swing/clash detection.  The board is not capable
of RGB color control at the present time, and only supports a single channel. Nonetheless, the system
could be modified to support RGB color without too much effort. The system is decidedly a high-end
system at the present time (costing approximately 200 USD per board), so a slightly less featured but
cheaper board could be a prominent competitor to this system.

\subsection[Patent]{Patent Analysis}
% Subsection: Patent Analysis
% Description: Analysis of project compared to existing patents
At the present time, the RGBSaber project uses simple PWM color mixing techniques. These techniques are
trivial and well documented in the public domain, and, as such, patent analysis and infringment is not
anticipated or likely.

%\section[Sources Cited]{Sources Cited}
% Section: Sources Cited
% Description: Citations for all sources used within the document.
%	Use of the IEEE citation style is highly recommended. For guidelines on using the IEEE citation style, 
%	refer to:
%	http://www.ieee.org/portal/cms_docs_iportals/iportals/publications/authors/transjnl/stylemanual.pdf

\end{document}
