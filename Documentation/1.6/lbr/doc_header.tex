% File: doc_header.tex
% Description: TeX command used to develop custom documentation header used in Open
%	 Documentation System
% Author: George Hadley
% Website: http://nbitwonder.com
% Notes: 
% 	1) Derived from LaTeX example 'TeXblog: Fancy chapter headings with TikZ' found at
%		http://texblog.net/latex-archive/layout/fancy-chapter-tikz/
%	2) For additional help with the LaTeX package, consult the WikiBook found at
%		http://en.wikibooks.org/wiki/LaTeX
%	3) This file needs to be compiled with pdfLaTeX (does not appear to be compatible with
%		pdfTeX at the present time)
% Version: 0.1
% Last Modified: 12-28-2010

%Variable Declarations (modify these to tweak the header on your document)
\newcommand{\headery}{-3cm}					%y-offset of header (bottom left corner of header)
\newcommand{\headercolor}{black}				%Background color of header
\newcommand{\headerxstart}{.1\paperwidth}			%Starting x-coordinate for header box
\newcommand{\headerystart}{0cm}				%Starting y-coordinate for header box
\newcommand{\headerxend}{.9\paperwidth}			%Ending x-coordinate for header box
\newcommand{\headeryend}{2cm}				%Ending y-coordinate for header box
\newcommand{\logoxstart}{.1\paperwidth}			%Starting x-coordinate for logo
\newcommand{\logoystart}{1cm}					%Starting y-coordinate for logo
\newcommand{\logolink}{http://nbitwonder.com}		%URL for website, etc.
\newcommand{\logoheight}{40pt}					%Height of logo image
\newcommand{\logoloc}{../lbr/img/NBitWonderLogo_wTm.png}	%Path to logo image
\newcommand{\docboxxstart}{.5\paperwidth}		%Starting x-coordinate of documentation box
\newcommand{\docboxystart}{1cm}				%Starting y-coordinate of documentation box
\newcommand{\docboxwidth}{.38\paperwidth}		%Width of documentation box
\newcommand{\docboxtextcolor}{green}			%Text color in documentation box

% Command: \nbwheader
% Description: Creates a documentation header for use in the Open Documentation System templates
% Usage: \nbwheader{DocumentName}{ProjectName}{ProjectVersion} where
%	DocumentName is the name of the document (e.g. Project Notebook, User Manual, Bug Tracker, etc.
%	ProjectName is the name of the project the documentation is about
%	ProjectVersion is the current version of the project
\newcommand{\nbwheader}[3]
{
    \begin{tikzpicture}[remember picture,overlay]
    \node[yshift=\headery] at (current page.north west)
    {
        \begin{tikzpicture}[remember picture, overlay]
        \draw[fill=\headercolor] (\headerxstart,\headerystart) rectangle
            (\headerxend,\headeryend);
        \node[anchor=west,xshift=\logoxstart,yshift=\logoystart,rectangle]
        {\href{\logolink}{\includegraphics[height=\logoheight]{\logoloc}}};
        \node[anchor=west,xshift =\docboxxstart,yshift=\docboxystart,rectangle]
        {
            \begin{minipage}{\docboxwidth}
	   \begin{flushright}
	   \normalsize\textcolor{\docboxtextcolor}{\textsc{
	   \linespread{2}
	   #1 \\
	   Project: #2 \\
	   Version: #3 \\
            }}
	   \end{flushright}
            \end{minipage}};
        \end{tikzpicture}
        };
    \end{tikzpicture}
}
